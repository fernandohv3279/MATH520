\documentclass{article}
\usepackage{amsmath}
\usepackage{mathtools}
\usepackage{amssymb}
\usepackage{graphicx}
\usepackage{enumitem}
\usepackage{listings}
\newcommand{\bld}[1]{\boldsymbol{#1}}

\begin{document}

\title{MATH520 Exam 1}
\author{Fernando}
\date{\today}
\maketitle

\section*{Problem 1}
\subsection*{Part a}
There are no feasible directions. This is easy to see geometrically because
$\Omega$ is a circle so any direction will take us out of the circle. Now let's
prove this formally by contradiction. Suppose that $d$ is a feasible direction
at $x^*$, then there exits $\alpha_0>0$ such that $x^*+\alpha d\in\Omega$ for
all $\alpha \in [0,\alpha_0]$. Notice that
\begin{align*}
||x^*+\alpha d||&=1\\
||x^*||+\alpha^2||d||^2+2\alpha (d\cdot x^*)&=1\\
\alpha^2||d||^2+2\alpha (d\cdot x^*)&=0,
\end{align*}
which is a quadratic equation on $\alpha$ so it has at most 2 solutions, which
contradicts the fact that $\alpha$ can be any value on $[0,\alpha_0]$.
\subsection*{Part b}
The FONC condition is trivially satisfied by every point in $\Omega$, because
in part a we saw that there are no feasible directions.
\subsection*{Part c}
It is not useful, it doesn't eliminate any point.
\subsection*{Part d}
If we define
\[
g(\theta)\coloneq f(\cos(\theta),\sin(\theta))
\]
then by the chain rule we get
\begin{align*}
g'(\theta)
&=Df(\cos(\theta),\sin(\theta))\cdot
[-\sin(\theta),\cos(\theta)]^T\\
&=[-\sin(\theta),\cos(\theta)]\cdot
\nabla f(\cos(\theta),\sin(\theta))
\end{align*}

Then if $x^*\in\Omega$ is a local minimizer there is a $\theta^*$ such
that\\
$[\cos(\theta^*),\sin(\theta^*)]=x^*$
and $g'(\theta^*)=0$ (unconstrained FONC).

So
\begin{align*}
0=
g'(\theta^*)
&=[-\sin(\theta^*),\cos(\theta^*)]\cdot
\nabla f(\cos(\theta^*),\sin(\theta^*))\\
&=[-\sin(\theta^*),\cos(\theta^*)]\cdot
\nabla f(x^*).
\end{align*}
Notice that
$[\cos(\theta^*),\sin(\theta^*)]\cdot[-\sin(\theta^*),\cos(\theta^*)]^T=0$, in
other words:\\
$[-\sin(\theta^*),\cos(\theta^*)]^T$ is a perpendicular vector to
$x^*$.

So we can say that if $x^*\in\Omega$ is a local minimizer, then $d^T\nabla
f(x^*)=0$ for all $d$ perpendicular to $x^*$.
\section*{Problem 2}
\subsection*{Part a}
\textbf{Lemma:} Under the conditions of the problem
\[
\alpha_k = -\frac{d^{(k)T}g^{(k)}}{d^{(k)T}Qd^{(k)}}
\]
\textbf{Proof:}

Let $h(\alpha)\coloneq f(x^{(k)}+\alpha_kd^{(k)})$. Then
\[
	h'(\alpha)=d^{(k)T}\nabla f(x^*+\alpha d^{(k)})
	=d^{(k)T}(Q(x^{(k)}+\alpha d^{(k)})-b).
\]
By the definition of $\alpha_k$ it has to satisfy the FONC
\[
	h'(\alpha_k)=0,
\]
so we get the equation
\[
0=d^{(k)T}(Q(x^{(k)}+\alpha_k d^{(k)})-b),
\]
and solving for $\alpha_k$ we get
\[
	\alpha_k=\frac{d^{(k)T}b-d^{(k)T}Qx^{(k)}}{d^{(k)T}Qd^{(k)}}=-\frac{d^{(k)T}g^{(k)}}{d^{(k)T}Qd^{(k)}},
\]
which concludes the proof of the lemma.\\
\textbf{Now let's do part a.}
\begin{align*}
2w_k
&= 2(V(x^{(k)})-V(x^{(k+1)}))\\
&= 2(V(x^{(k)})-V(x^{(k+1)}))\\
&=(x^{(k)}-x^*)^TQ(x^{(k)}-x^*)
-(x^{(k+1)}-x^*)^TQ(x^{(k+1)}-x^*)\\
&=x^{(k)T}Qx^{(k)}-x^{(k)T}Qx^*-x^{*T}Qx^{(k)}+x^{*T}Qx^*\\
&\quad-x^{(k+1)T}Qx^{(k+1)}+x^{(k+1)T}Qx^*+x^{*T}Qx^{(k+1)}-x^{*T}Qx^*\\
\end{align*}
replacing $x^{(k+1)}=x^{(k)}+\alpha_k d^{(k)}$ and simplifying we get that
\begin{align*}
2w_k&= \alpha_k d^{(k)T}Qx^*+\alpha_k x^{*T}Qd^{(k)}
-\alpha_k d^{(k)T}Qx^{(k)}-\alpha_k x^{(k)T}Qd^{(k)}\\
&-\alpha_k^2d^{(k)T}Qd^{(k)}\\
&=2\alpha_k d^{(k)T}Qx^*-2\alpha_k d^{(k)T}Qx^{(k)}-\alpha_k^2d^{(k)T}Qd^{(k)}\\
&=-2\alpha_k d^{(k)T}g^{(k)}-\alpha_k^2d^{(k)T}Qd^{(k)}\\
&=-2\alpha_k d^{(k)T}g^{(k)}-\alpha_k^2d^{(k)T}Qd^{(k)}\\
&=\frac{(d^{(k)T}g^{(k)})^2}{d^{(k)T}Qd^{(k)}} \text{ (by the previous lemma).}
\end{align*}
\subsection*{Part b}
Suppose that $x^{(k)}\neq x^*$ and consider
\[
r_k=\frac{V(x^{(k)}) -V(x^{(k+1)})}{V(x^{(k)})}
=\frac{\frac{\left(d^{(k)T}g^{(k)}\right)^2}{d^{(k)T}Qd^{(k)}}}{(x^{(k)}-x^*)^TQ(x^{(k)}-x^*)},
\]
notice that $r_k\geq0$ because $Q$ is positive definite, also $r_k\leq 1$ because\\
$r_k=1-V(x^{(k+1)})/V(x^{(k)})$ and $V\geq 0$, then
\[
V(x^{(k+1)})=(1-r_k)V(x^{(k)}).
\]
For the case $x^{(k)}=x^*$ the equation above is satisfied for any $r_k$
because $V(x^{(k)})=V(x^{(k+1)})=0$, so let's pick $r_k=1$ for this case (we
could have picked any other number with norm at most one). Now we have defined
$r_k$ properly and we can write
\[
V(x^{(k+1)})=\prod_{i=0}^k(1-r_i)V(x^{(0)})=V(x^{(0)})\prod_{i=0}^k(1-r_i).
\]
In order to analyse $\lim_{k\to\infty}V(x^{(k+1)})$ we consider two cases.

\textbf{Case 1:} There is an infinite number of positive $r_i$'s. In this case
the limit
\[
	\lim_{k\to\infty}V(x^{(k+1)})=V(x^{(0)})\lim_{k\to\infty}\prod_{i=0}^k(1-r_i)
\]
is convergent because we have a strictly decreasing sequence bounded below (by
0).

\textbf{Case 2:} There is a finite number of positive $r_i$'s. In this case at
some point all the $r_i$'s become 0 and we are left with a tail of ones in the
product which again converges.

\textbf{Conclusion:} $\lim_{k\to\infty}V(x^{(k+1)})$ is convergent.\\
\textbf{Now let's finish part b}

From part a we get
\[
V(x^{(k+1)})=V(x^{(0)})-\sum_{i=0}^kw_i,
\]
so
\[
\sum_{i=0}^kw_i=V(x^{(0)})-V(x^{(k+1)})
\]
taking the limit as $k\to\infty$ and using the fact that
$\lim_{k\to\infty}V(x^{(k+1)})$ is convergent we get the result.
\subsection*{Part c}
From the formula given we have
\[
||g^{(k)}||^2\cos^2(\theta_k)=\frac{(d^{(k)T}g^{(k)})^2}{||d^{(k)}||^2}
\]
and by Rayleigh's inequality we get
\[
||g^{(k)}||^2\cos^2(\theta_k)=\frac{(d^{(k)T}g^{(k)})^2}{||d^{(k)}||^2}
\leq2\lambda_{\max}(Q)w_k,
\]
where $w_k$ was obtained in part a.

Taking the sum on both sides and using part b yields the result.
\subsection*{Part d}
\[
	\sum_{k=0}^\infty \delta^2||g^{(k)}||^2\leq
	\sum_{k=0}^\infty \cos^2(\theta_k)||g^{(k)}||^2
	<\infty,
\]
which implies that $||g^{(k)}||\to 0$, then
\[
	||x^{(k)}-x^*||=||Q^{-1}(Qx^{(k)}-b)||\leq||Q^{-1}||\cdot||g^{(k)}||
\]
Taking the limit as $k\to \infty$ we obtain the result.
\section*{Problem 3}
We proceed by induction as suggested by the hint.

For $k=0$ it is clear that $x^{(0)}=0\in\{0\}=\mathcal{V}_0$ and
$d^{(0)}=-g^{(0)}=b\in \text{span}[b]=\mathcal{V}_1$.

For the induction step we assume that the result is true for $k$ and we have to
prove that $x^{(k+1)}\in \mathcal{V}_{k+1}$ and $d^{(k+1)}\in
\mathcal{V}_{k+2}$.

Since
\[
x^{(k+1)}=x^{(k)}+\alpha_kd^{(k)}
\]
and by I.H. $x^{(k)}\in \mathcal{V}_k$ and $d^{(k)}\in\mathcal{V}_{k+1}$ then
$x^{(k+1)}\in
\mathcal{V}_{k+1}$ (Clearly $\mathcal{V}_k\subset\mathcal{V}_{k+1}$).

As for $d^{(k+1)}$ we have
\[
d^{(k+1)}=-g^{(k+1)}+\beta_k d^{(k)},
\]
since $d^{(k)}\in\mathcal{V}_{k+1}$ by I.H. it is enough to prove that
$g^{(k+1)}\in\mathcal{V}_{k+2}$ (because
$\mathcal{V}_{k+1}\subset\mathcal{V}_{k+2}$). To do this notice that
$x^{(k+1)}$ is a linear combination of the vectors $b,Qb,\dots,Q^{k}b$, then
\begin{align*}
g^{(k+1)}=Qx^{(k+1)}-b&=Q\left(\sum_{i=0}^kc_iQ^ib\right) -b\\
&=\left(\sum_{i=0}^kc_iQ^{i+1}b\right)-b \in
\text{span}[b,Qb,\dots,Q^{k+1}b]=\mathcal{V}_{k+2},
\end{align*}
which concludes the proof.
\end{document}
