\documentclass{article}
\usepackage{amsmath}
\usepackage{mathtools}
\usepackage{amssymb}
\usepackage{graphicx}
\usepackage{enumitem}
\usepackage{listings}
\newcommand{\bld}[1]{\boldsymbol{#1}}

\begin{document}

\title{MATH520 Exam 1}
\author{Fernando}
\date{\today}
\maketitle

\section*{Problem 1}
\subsection*{Part a}
There are no feasible directions. This is easy to see geometrically because
$\Omega$ is a circle so any direction will take us out of the circle. Now let's
prove this formally by contradiction. Suppose that $d$ is a feasible direction
at $x^*$, then there exits $\alpha_0>0$ such that $x^*+\alpha d\in\Omega$ for
all $\alpha \in [0,\alpha_0]$. Notice that
\begin{align*}
||x^*+\alpha d||&=1\\
||x^*||+\alpha^2||d||^2+2\alpha (d\cdot x^*)&=1\\
\alpha^2||d||^2+2\alpha (d\cdot x^*)&=0,
\end{align*}
which is a quadratic equation on $\alpha$ so it has at most 2 solutions, which
contradicts the fact that $\alpha$ can be any value on $[0,\alpha_0]$.
\subsection*{Part b}
The FONC condition is trivially satisfied by every point in $\Omega$, because
in part a we saw that there are no feasible directions.
\subsection*{Part c}
It is not useful, it doesn't eliminate any point.
\subsection*{Part d}
If we define
\[
g(\theta)\coloneq f(\cos(\theta),\sin(\theta))
\]
then by the chain rule we get
\begin{align*}
g'(\theta)
&=Df(\cos(\theta),\sin(\theta))\cdot
[-\sin(\theta),\cos(\theta)]^T\\
&=[-\sin(\theta),\cos(\theta)]\cdot
\nabla f(\cos(\theta),\sin(\theta))
\end{align*}

Then if $x^*\in\Omega$ is a local minimizer then there is a $\theta^*$ such
that\\
$[\cos(\theta^*),\sin(\theta^*)]=x^*$
and $g'(\theta^*)=0$ (unconstrained FONC).

So
\begin{align*}
0=
g'(\theta^*)
&=[-\sin(\theta^*),\cos(\theta^*)]\cdot
\nabla f(\cos(\theta^*),\sin(\theta^*))\\
&=[-\sin(\theta^*),\cos(\theta^*)]\cdot
\nabla f(x^*).
\end{align*}
Notice that
$[\cos(\theta^*),\sin(\theta^*)]\cdot[-\sin(\theta^*),\cos(\theta^*)]^T=0$, in
other words:\\
$[-\sin(\theta^*),\cos(\theta^*)]^T$ is the perpendicular vector to
$x^*$.

So we can say that if $x^*\in\Omega$ is a local minimizer, then $d^T\nabla
f(x^*)=0$ for all $d$ perpendicular to $x^*$.
\section*{Problem 2}
\subsection*{Part a}
\subsection*{Part b}
\subsection*{Part c}
\subsection*{Part d}
\section*{Problem 3}
We proceed by induction as suggested by the hint.

For $k=0$ it is clear that $x^{(0)}=0\in\{0\}=\mathcal{V}_0$ and
$d^{(0)}=-g^{(0)}=b\in \text{span}[b]=\mathcal{V}_1$.

For the induction step we assume that the result is true for $k$ and we have to
prove that $x^{(k+1)}\in \mathcal{V}_{k+1}$ and $d^{(k+1)}\in
\mathcal{V}_{k+2}$.

Since
\[
x^{(k+1)}=x^{(k)}+\alpha_kd^{(k)}
\]
and by I.H. $x^{(k)}\in \mathcal{V}_k$ and $d^{(k)}\in\mathcal{V}_{k+1}$ then
$x^{(k+1)}\in
\mathcal{V}_{k+1}$ (Clearly $\mathcal{V}_k\subset\mathcal{V}_{k+1}$).

As for $d^{(k+1)}$ we have
\[
d^{(k+1)}=-g^{(k+1)}+\beta_k d^{(k)},
\]
since $d^{(k)}\in\mathcal{V}_{k+1}$ by I.H. it is enough to prove that
$g^{(k+1)}\in\mathcal{V}_{k+2}$ (because
$\mathcal{V}_{k+1}\subset\mathcal{V}_{k+2}$). To do this notice that
$x^{(k+1)}$ is a linear combination of the vectors $b,Qb,\dots,Q^{k}b$, then
\begin{align*}
g^{(k+1)}=Qx^{(k+1)}-b&=Q\left(\sum_{i=0}^kc_iQ^ib\right) -b\\
&=\left(\sum_{i=0}^kc_iQ^{i+1}b\right)-b \in
\text{span}[b,Qb,\dots,Q^{k+1}b]=\mathcal{V}_{k+2},
\end{align*}
which concludes the proof.
\end{document}
