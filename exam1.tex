\documentclass{article}
\usepackage{amsmath}
\usepackage{amssymb}
\usepackage{graphicx}
\usepackage{enumitem}
\usepackage{listings}
\newcommand{\bld}[1]{\boldsymbol{#1}}

\begin{document}

\title{MATH520 Exam 1}
\author{Fernando}
\date{\today}
\maketitle

\section*{Problem 1}
\section*{Problem 2}
\section*{Problem 3}
We proceed by induction as suggested by the hint.

For $k=0$ it is clear that $x^{(0)}=0\in\{0\}=\mathcal{V}_0$ and
$d^{(0)}=-g^{(0)}=b\in \text{span}[b]=\mathcal{V}_1$.

For the induction step we assume that the result is true for $k$ and we have to
prove that $x^{(k+1)}\in \mathcal{V}_{k+1}$ and $d^{(k+1)}\in
\mathcal{V}_{k+2}$.

Since
\[
x^{(k+1)}=x^{(k)}+\alpha_kd^{(k)}
\]
and by I.H. $x^{(k)}\in \mathcal{V}_k$ and $d^{(k)}\in\mathcal{V}_{k+1}$ then
$x^{(k+1)}\in
\mathcal{V}_{k+1}$ (Clearly $\mathcal{V}_k\subset\mathcal{V}_{k+1}$).

As for $d^{(k+1)}$ we have
\[
d^{(k+1)}=-g^{(k+1)}+\beta_k d^{(k)},
\]
since $d^{(k)}\in\mathcal{V}_{k+1}$ by I.H. it is enough to prove that
$g^{(k+1)}\in\mathcal{V}_{k+2}$ (because
$\mathcal{V}_{k+1}\subset\mathcal{V}_{k+2}$). To do this notice that
$x^{(k+1)}$ is a linear combination of the vectors $b,Qb,\dots,Q^{k}b$, then
\begin{align*}
g^{(k+1)}=Qx^{(k+1)}-b&=Q\left(\sum_{i=0}^kc_iQ^ib\right) -b\\
&=\left(\sum_{i=0}^kc_iQ^{i+1}b\right)-b \in
\text{span}[b,Qb,\dots,Q^{k+1}b]=\mathcal{V}_{k+2},
\end{align*}
which concludes the proof.
\end{document}
