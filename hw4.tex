\documentclass{article}
\usepackage{amsmath}
\usepackage{amssymb}
\usepackage{graphicx}
\usepackage{enumitem}
\newcommand{\bld}[1]{\boldsymbol{#1}}

\begin{document}

\title{MATH520 Homework 3}
\author{Fernando}
\date{\today}
\maketitle

\section*{Exercise 10.1}
\subsection*{Part 1: The algorithm is well defined}
It is enough to prove that $d^{(i)T}Qd^{(i)}\neq 0$ for all $i$.
We proceed by induction:
\subsection*{Part 2: $d^{(0)},\dots,d^{(n-1)}$ are Q-conjugate}
We can use induction over $k$. For $k=1$ we have
\[
	d^{(1)}=p^{(1)}-\frac{p^{(1)T}Qd^{(0)}}{d^{(0)T}Qd^{(0)}}d^{(0)},
\]
and then:
\begin{align*}
	d^{(0)T}Qd^{(1)}&=
	d^{(0)T}Q\left(p^{(1)}-\frac{p^{(1)T}Qd^{(0)}}{d^{(0)T}Qd^{(0)}}d^{(0)}\right)\\
	&=
	d^{(0)T}Qp^{(1)}-p^{(1)T}Qd^{(0)} \text{ (by I.H.)}\\
	&=0.
\end{align*}
Now assuming the result is true for $k$ we prove it for $k+1$. For
$j=1,\dots,k$ we have:
\begin{align*}
	d^{(i)T}Qd^{(k+1)}&=
	d^{(j)T}Q\left(p^{(k+1)}-\sum_{i=0}^k\frac{p^{(k+1)T}Qd^{(i)}}{d^{(i)T}Qd^{(i)}}d^{(i)}\right)\\
	&=
	d^{(j)T}Qp^{(k+1)}-p^{(k+1)T}Qd^{(j)} \text{ (by I.H.)}\\
	&=0.
\end{align*}
\section*{Exercise 10.7}
\section*{Exercise 10.10}
\end{document}
