\documentclass{article}
\usepackage{amsmath}
\usepackage{amssymb}
\usepackage{graphicx}
\usepackage{enumitem}
\usepackage{listings}
\newcommand{\bld}[1]{\boldsymbol{#1}}

\begin{document}

\title{MATH520 Homework 4}
\author{Fernando}
\date{\today}
\maketitle

\section*{Exercise 10.1}
\subsection*{Part 1: The algorithm is well defined}
It is enough to prove that $d^{(i)T}Qd^{(i)}\neq 0$ for all $i$. Since $Q\succ
0$ it is enough to show that $d^{(i)}\neq 0$.
We proceed by induction. For the base case we have: $d^{(0)}=p^{(0)}$ which is
clearly not 0. For the induction step if we had:
\[
0=d^{(k+1)}=
\left(p^{(k+1)}-\sum_{i=0}^k\frac{p^{(k+1)T}Qd^{(i)}}{d^{(i)T}Qd^{(i)}}d^{(i)}\right)
\]
Then that implies
\[
p^{(k+1)}=\sum_{i=0}^k\frac{p^{(k+1)T}Qd^{(i)}}{d^{(i)T}Qd^{(i)}}d^{(i)},
\]
or in other words $p^{(k+1)}\in$ span($d^{(0)},\dots,d^{(k)}$), but notice that
by construction $d^{(i)}\in$ span($p^{(0)},\dots,p^{(i)}$), then $p^{(k+1)}\in$
span($p^{(0)},\dots,p^{(k)}$) which is a contradiction!

\textbf{conclusion:} $d^{(k+1)}\neq 0$, and that finishes the proof.
\subsection*{Part 2: $d^{(0)},\dots,d^{(n-1)}$ are Q-conjugate}
We can use induction over $k$. For $k=1$ we have
\[
	d^{(1)}=p^{(1)}-\frac{p^{(1)T}Qd^{(0)}}{d^{(0)T}Qd^{(0)}}d^{(0)},
\]
and then:
\begin{align*}
	d^{(0)T}Qd^{(1)}&=
	d^{(0)T}Q\left(p^{(1)}-\frac{p^{(1)T}Qd^{(0)}}{d^{(0)T}Qd^{(0)}}d^{(0)}\right)\\
	&=
	d^{(0)T}Qp^{(1)}-p^{(1)T}Qd^{(0)} \text{ (by I.H.)}\\
	&=0.
\end{align*}
Now assuming the result is true for $k$ we prove it for $k+1$. For
$j=1,\dots,k$ we have:
\begin{align*}
	d^{(i)T}Qd^{(k+1)}&=
	d^{(j)T}Q\left(p^{(k+1)}-\sum_{i=0}^k\frac{p^{(k+1)T}Qd^{(i)}}{d^{(i)T}Qd^{(i)}}d^{(i)}\right)\\
	&=
	d^{(j)T}Qp^{(k+1)}-p^{(k+1)T}Qd^{(j)} \text{ (by I.H.)}\\
	&=0.
\end{align*}
\section*{Exercise 10.7}
\begin{align*}
	\phi(a)&=\frac{1}{2}(x_0+Da)^TQ(x_0+Da)-(x_0+Da)^Tb\\
	&=\frac{1}{2}x_0^TQx_0+\frac{1}{2}x_0^TQDa+\frac{1}{2}(Da)^TQx_0+\frac{1}{2}(Da)^TQDa -x_0^Tb-(Da)^Tb\\
	&=\frac{1}{2}a^TD^TQx_0+\frac{1}{2}a^TD^TQDa-a^TD^Tb+\frac{1}{2}x_0^TQDa + \frac{1}{2}x_0^TQx_0-x_0^Tb\\
	&=\frac{1}{2}a^TD^TQDa-a^TD^Tb+\frac{1}{2}a^TD^TQx_0+\frac{1}{2}a^TD^TQx_0 + \frac{1}{2}x_0^TQx_0-x_0^Tb\\
	&=\frac{1}{2}a^TD^TQDa-a^TD^Tb+a^TD^TQx_0+ \frac{1}{2}x_0^TQx_0-x_0^Tb\\
	&=\frac{1}{2}a^T\tilde{Q}a^T+a^T\tilde{b}+\tilde{c}
\end{align*}
Where: $\tilde{Q}=D^TQD$, $\tilde{b}=D^TQx_0-D^Tb$, $\tilde{c}=\frac{1}{2}x_0^TQx_0-x_0^Tb$.

Now we need to prove that $\tilde{Q}$ is positive definite (notice that it is symmetric).

For this if we take $x\in\mathbb{R}^r$ we get that
\[
	x^TD^TQDx=(Dx)^TQ(Dx)\geq 0.
\]
So $\tilde{Q}$ is definitely positive semi-definite. For $\tilde{Q}$ to be
positive definite we need to ensure that $Dx\neq 0$. Which is true if $r\leq n$
but false if $r>n$ (rank-nullity theorem).
\section*{Exercise 10.10}
\subsection*{Part a}
By comparing terms we get:
\[
	f(x)=\frac{1}{2}x^T\begin{bmatrix}
	5 & 2 \\
	2 & 1\\
	\end{bmatrix}x
	- x^T\begin{bmatrix}
	3\\
	1
	\end{bmatrix}.
\]
\subsection*{Part b}
\subsubsection*{First attempt}
\begin{align*}
	g^{(0)}=&\nabla f([0,0]^T)=-[3,1]^T\\
	d^{(0)}=&[3,1]^T\\
	\alpha_0=&-\frac{g^{(0)T}d^{(0)}}
	{d^{(0)T}Qd^{(0)}}=\frac{10}{58}\\
	x^{(1)}=&\frac{10}{58}\cdot[3,1]^T=[30/58,10/58]^T\\
	g^{(1)}=&\nabla f(x^{(1)})=\frac{2}{29}[-1,3]^T\\
	\beta_0=&\frac{g^{(1)T}Qd^{(0)}}
	{d^{(0)T}Qd^{(0)}}=\frac{8/29}{58}\\
	d^{(1)}=&-[-2/29,6/29]^T+\frac{8/29}{58}\cdot[3,1]^T\\
	    =&[70/841,-170/841]^T\\
	\alpha_1=&-\frac{g^{(1)T}d^{(1)}}
	{d^{(1)T}Qd^{(1)}}=\frac{-40/841}{200/24389}=\frac{29}{5}\\
	x^{(2)}=&[30/58,10/58]^T-\frac{29}{5}[70/841,-170/841]^T\\
	=&[1/29,39/29]\\
	g^{(2)}=&[-4/29,12/29]
\end{align*}
There is a mistake in the calculation but I can't find it, this computation is
too tedious.
\subsubsection*{Second attempt}
We can solve it with the following python code:
\begin{lstlisting}
import numpy as np

Q=np.array([[5,2],[2,1]])
def grad(vector):
    return np.matmul(Q,vector)-np.array([[3],[1]])

def cgm():
    xk=np.array([[0],[0]])
    gk=grad(xk)
    print(gk)
    dk=-1*gk
    tol=1e-15
    maxiter=100
    for k in range(maxiter):
        print("iter " + str(k))
        ak=-1*(np.matmul(np.transpose(gk),dk))/\
        (np.matmul(np.matmul(np.transpose(dk),Q),dk))
        print("ak")
        print(ak)
        xk1=xk+ak*dk
        print("xk1")
        print(xk1)
        gk1=grad(xk1)
        print("gk1")
        print(gk1)
        if(np.linalg.norm(gk1)<tol):
            print("Found it!")
            print(xk1)
            return
        bk=\
        (np.matmul(np.matmul(np.transpose(gk1),Q),dk))/\
        (np.matmul(np.matmul(np.transpose(dk),Q),dk))
        print("bk")
        print(bk)
        dk1=-1*gk1+bk*dk
        print("dk1")
        print(dk1)
        # update stuff
        xk=xk1
        gk=gk1
        dk=dk1
cgm()
\end{lstlisting}
Which produces the following output:
\begin{lstlisting}
[[-3]
 [-1]]
iter 0
ak
[[0.17241379]]
xk1
[[0.51724138]
 [0.17241379]]
gk1
[[-0.06896552]
 [ 0.20689655]]
bk
[[0.00475624]]
dk1
[[ 0.08323424]
 [-0.20214031]]
iter 1
ak
[[5.8]]
xk1
[[ 1.]
 [-1.]]
gk1
[[8.8817842e-16]
 [4.4408921e-16]]
Found it!
[[ 1.]
 [-1.]]
\end{lstlisting}
So the vector is $[1,-1]^T$ which is found on the second iteration as expected.
\subsection*{Part c}
\[
\nabla f(x)=\begin{bmatrix}
	5 & 2 \\
	2 & 1\\
	\end{bmatrix}x
	- \begin{bmatrix}
	3\\
	1
	\end{bmatrix},
\]
then
\[
	x=\begin{bmatrix}
	1\\
	-1\\
	\end{bmatrix}.
\]
Which coincides with the previous part.
\end{document}
