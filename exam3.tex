\documentclass{article}
\usepackage{enumerate}
\usepackage{amsmath}
\usepackage{mathtools}
\usepackage{amssymb}
\usepackage{graphicx}
% \usepackage{enumitem}
\usepackage{listings}
\newcommand{\bld}[1]{\boldsymbol{#1}}

\begin{document}

\title{MATH520 Exam 3}
\author{Fernando}
\date{\today}
\maketitle

\section*{Problem 1}
\subsection*{Part a}
\subsection*{Part b}
\subsection*{Part c}
\subsection*{Part d}
\section*{Problem 2}
\subsection*{Part i}
This is possible. An example is given in the book. Take as the primal
\[
\begin{aligned}
\text{minimize}\quad & x\\
\textrm{s.t.} \quad &x\leq 1,\\
\end{aligned}
\]
then the dual is
\[
\begin{aligned}
\text{maximize}\quad & \lambda\\
\textrm{s.t.} \quad &\lambda= 1\\
\quad &\lambda\leq 0.\\
\end{aligned}
\]
Clearly the primal has a feasible solution and the dual does not.
\subsection*{Part ii}
This is impossible by the duality theorem (if one has an optimal feasible
solution, then so does the other).
\subsection*{Part iii}
This is impossible by the duality theorem (if one has an optimal feasible
solution, then so does the other).
\subsection*{Part iv}
This is possible. An example is given in the book. Take as the primal
\[
\begin{aligned}
	\text{minimize}\quad & [1,-2]x\\
\textrm{s.t.} \quad &\begin{bmatrix}
	1 & -1\\
	-1 & 1
\end{bmatrix}x\geq \begin{bmatrix}
2\\	
-1
\end{bmatrix}.\\
	      \quad &x\geq 0.\\
\end{aligned}
\]
Which has no feasible solution because it requires $[1,-1]x\geq 2$ and
$[1,-1]x\leq 1$. The dual of this problem is
\[
\begin{aligned}
	\text{maximize}\quad & \lambda^T\begin{bmatrix}
	2\\	
	-1
	\end{bmatrix}\\
\textrm{s.t.} \quad &\lambda^T\begin{bmatrix}
	1 & -1\\
	-1 & 1
	\end{bmatrix}\leq [1,-2]\\
	      \quad &\lambda\geq 0.\\
\end{aligned}
\]
Which again has no feasible solution because it requires $\lambda^T\begin{bmatrix}
1\\
-1
\end{bmatrix}\leq 1$ and $\lambda^T\begin{bmatrix}
1\\
-1
\end{bmatrix}\geq 2$.
\section*{Problem 3}
\subsection*{Part a}
Our problem is the following
\[
\begin{aligned}
\text{minimize}\quad & f(x)\\
\textrm{s.t.} \quad &g(x)\leq 0.\\
\end{aligned}
\]
Where $f(x)=x_1x_2$ and $g(x)=\begin{bmatrix}
g_1(x)\\
g_2(x)
\end{bmatrix}=\begin{bmatrix}
2-x_1-x_2\\
x_1-x_2
\end{bmatrix}$

Since in this problem we only have inequality constraints, the KKT condition
is:
\begin{enumerate}[I]
\item $\mu^*\geq0$ \label{muPositive}
\item $Df(x^*)+{\mu^*}^TDg(x^*)=0^T$ \label{KKT}
\item ${\mu^*}^Tg(x^*)=0$ \label{muTg}
\item $g(x^*)\leq0$ \label{gNegative}
\end{enumerate}
\subsection*{Part b}
Notice that condition \ref{muTg} can be written as
\begin{equation} \label{cond3reWrite}
\mu_1g_1 + \mu_2g_2=0.
\end{equation}
Because of conditions \ref{muPositive} and \ref{gNegative} each of the
terms in (\ref{cond3reWrite}) must be 0, so $\mu_ig_i=0$.
If we suppose $\mu_1=0$ condition \ref{KKT} gives us the following:
\begin{align*}
0^T&=Df(x^*)+{\mu^*}^TDg(x^*)\\
&=[x_2,x_1]+[\mu_1,\mu_2]\begin{bmatrix}
-1 & -1\\
1 & -1
\end{bmatrix}\\
&=[x_2,x_1]+[0,\mu_2]\begin{bmatrix}
-1 & -1\\
1 & -1
\end{bmatrix}\\
&=[ x_2+\mu_2, x_1-\mu_2].
\end{align*}
Adding these two equations we get
\[
x_1+x_2=0,
\]
but this is a contradiction with condition \ref{gNegative}, in particular it
contradicts $g_1(x)=2-x_1-x_2\leq0$. Conclusion: $u_1\neq 0$, so we know that
for sure $g_1=0$.
With regards to the other term in (\ref{cond3reWrite}), if we assume
$g_2(x)=0$, this means $x_1=x_2$ and replacing this in \ref{KKT} we
get
\begin{align*}
0^T&=Df(x^*)+{\mu^*}^TDg(x^*)\\
&=[x_1,x_1]+[\mu_1,\mu_2]\begin{bmatrix}
-1 & -1\\
1 & -1
\end{bmatrix}\\
&=[ x_1-\mu_1+\mu_2, x_1-\mu_1-\mu_2].
\end{align*}
Again adding these equations we get $\mu_1=x_1$, and then $\mu_2=0$.
Conclusion: we know that for sure $\mu_2=0$. Once again using
\ref{KKT} we get:
\begin{align*}
0^T&=Df(x^*)+{\mu^*}^TDg(x^*)\\
&=[x_1,x_2]+[\mu_1,\mu_2]\begin{bmatrix}
-1 & -1\\
1 & -1
\end{bmatrix}\\
&=[x_1,x_2]+[\mu_1,0]\begin{bmatrix}
-1 & -1\\
1 & -1
\end{bmatrix}\\
&=[ x_1-\mu_1, x_2-\mu_1].
\end{align*}
From where $x_1=x_2=\mu_1$ and using the fact that $g_1=0$ we get that
$x_1=x_2=\mu_1=1$.

So the only point that satisfies the KKT condition is ${x^*}^T=[1,1]$, with the
KKT multiplier vector ${\mu^*}^T=[1,0]$.

Since $\nabla g_1(x^*)=\begin{bmatrix} -1\\-1\end{bmatrix}$ and $\nabla
g_2(x^*)=\begin{bmatrix} 1\\-1\end{bmatrix}$ are L.I. $x^*$ is regular.
\subsection*{Part c}
First we need to determine $T(x^*)$. Notice that both constrains are active so
\[
	T(x^*)=\{y\in \mathbb{R}^2:Dg(x^*)y=0\}
\]
and since $Dg(x^*)$ is invertible we have
\[
	T(x^*)=\{0\},
\]
so the SONC is trivially satisfied.
\subsection*{Part d}
In this case we have
\[
	\tilde{J}(x^*,\mu^*)=\{1\},
\]
and thus
\[
	\tilde{T}(x^*,\mu^*)=\{y:[-1,-1]y=0\}=\text{span}([1,-1]^T).
\]
Now we compute $F(x)$ and see that
\[
	F(x)=\begin{bmatrix}
		0 & 1\\	
		1 & 0
	\end{bmatrix},
\]
also notice that because $g_1$ and $g_2$ are linear in each variable we have
\[
G_1(x)=G_2(x)=\begin{bmatrix}
	0 & 0\\	
	0 & 0
\end{bmatrix},
\]
then
\[
L(x,\mu)= F(x)=\begin{bmatrix}
		0 & 1\\	
		1 & 0
	\end{bmatrix},
\]
and notice that $[1,-1]^T\in\text{span}([1,-1]^T)=\tilde{T}(x^*,\mu^*)$, and
\[
	[1,-1]L(x^*,\mu^*)\begin{bmatrix}
1\\	
-1
\end{bmatrix}=
[1,-1]
\begin{bmatrix}
0 & 1\\	
1 & 0
\end{bmatrix}
\begin{bmatrix}
1\\	
-1
\end{bmatrix}=
[-1,1]\begin{bmatrix}
1\\	
-1
\end{bmatrix}=-2<0.
\]
So no point in part c satisfies the SOSC.
\end{document}
