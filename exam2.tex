\documentclass{article}
\usepackage{amsmath}
\usepackage{mathtools}
\usepackage{amssymb}
\usepackage{graphicx}
\usepackage{enumitem}
\usepackage{listings}
\newcommand{\bld}[1]{\boldsymbol{#1}}

\begin{document}

\title{MATH520 Exam 2}
\author{Fernando}
\date{\today}
\maketitle

\section*{Problem 1}
\subsection*{Part a}
\[
	x^{(0)}=\left(A_0^TA_0\right)^{-1}A_0^Tb^{(0)}=G_0^{-1}A_0^Tb^{(0)}.
\]
\[
	x^{(1)}=\left(A_1^TA_1\right)^{-1}A_1^Tb^{(1)}=G_1^{-1}A_1^Tb^{(1)}.
\]
\subsection*{Part b}
Notice that
\begin{align*}
G_0=A_0^TA_0=\begin{bmatrix}
A_1\\
a_1^T
\end{bmatrix}^T
\begin{bmatrix}
A_1\\
a_1^T
\end{bmatrix}
=
\begin{bmatrix}
A_1^T & a_1
\end{bmatrix}
\begin{bmatrix}
A_1\\
a_1^T
\end{bmatrix}
=
A_1^TA_1+a_1a_1^T
=G_1+a_1a_1^T.
\end{align*}
So
\[
	G_1=G_0-a_1a_1^T.
\]
\subsection*{Part c}
Using the Sherman-Morrison formula with $A=G_0,u=-a_1$ and $v=a_1$ (we are
assuming $1-a_1^TG_0^{-1}a_1\neq 0$) we get
\[
	P_1=P_0-\frac{(P_0(-a_1))(a_1^TP_0)}{1-a_1^TG_0^{-1}a_1}=P_0+\frac{(P_0a_1)(a_1^TP_0)}{1-a_1^TP_0a_1}.
\]
\subsection*{Part d}
\begin{align*}
	A_0^Tb^{(0)}&=G_0G_0^{-1}A_0^Tb^{(0)}\\
	&=G_0x^{(0)}\\
	&=(G_1+a_1a_1^T)x^{(0)}\\
	&=G_1x^{(0)}+a_1a_1^Tx^{(0)}.
\end{align*}
\subsection*{Part e}
First we notice that
\begin{align*}
	A_0^Tb^{(0)}=\begin{bmatrix}
	A_1\\	
	a_1
	\end{bmatrix}^T
	\begin{bmatrix}
		b^{(1)}\\
		b_1
	\end{bmatrix}
	=
	A_1^Tb^{(1)}
\end{align*}
And because of the previous question:
\begin{align*}
	G_1x^{(0)}+a_1a_1^Tx^{(0)}=
	A_0^Tb^{(0)}=\begin{bmatrix}
	A_1\\	
	a_1
	\end{bmatrix}^T
	\begin{bmatrix}
		b^{(1)}\\
		b_1
	\end{bmatrix}
	=
	A_1^Tb^{(1)}+a_1b_1,
\end{align*}
then
\begin{align*}
A_1^Tb^{(1)} = G_1x^{(0)}+a_1a_1^Tx^{(0)}-a_1b_1=G_1x^{(0)}+a_1(a_1^Tx^{(0)}-b_1).
\end{align*}
Using this last expression we get
\begin{align*}
	x^{(1)}=G_1^{-1}A_1^Tb^{(1)}=P_1A_1^Tb^{(1)}=x^{(0)}+P_1a_1(a_1^Tx^{(0)}-b_1).
\end{align*}
Generalizing this we get
\begin{align*}
	P_{k+1}&=P_k+\frac{(P_ka_{k+1})(a_{k+1}^TP_k)}{1-a_{k+1}^TP_ka_{k+1}}\\
	x^{(k+1)}&=x^{(k)}+P_{k+1}a_{k+1}(a_{k+1}^Tx^{(k)}-b_{k+1}).
\end{align*}
\section*{Problem 2}
\subsection*{Part a}
\[
\begin{aligned}
\text{minimize}\quad & x_1+2x_2+3x_3+4x_4\\
\textrm{s.t.} \quad &x_1+x_3=50,x_2+x_4=60\\
  &x_1+x_2\leq70,x_3+x_4\leq80\\
  &x_1,x_2,x_3,x_4\geq0\\
\end{aligned}
\]
\subsection*{Part b}
\[
\begin{aligned}
\text{minimize}\quad & x_1+2x_2+3x_3+4x_4\\
\textrm{s.t.} \quad &x_1+x_3=50,x_2+x_4=60,x_1+x_2+x_5=70,x_3+x_4+x_6=80\\
  &x_1,x_2,x_3,x_4,x_5,x_6\geq0,\\
\end{aligned}
\]
which is equivalent to
\[
\begin{aligned}
\text{minimize}\quad & c^Tx\\
\textrm{s.t.} \quad &Ax=b\\
  &x\geq0,\\
\end{aligned}
\]
with $c=\begin{bmatrix}
1\\2\\3\\4\\0\\0
\end{bmatrix}$, $A=\begin{bmatrix}
1 & 0 & 1 & 0 & 0 & 0\\
0 & 1 & 0 & 1 & 0 & 0\\
1 & 1 & 0 & 0 & 1 & 0\\
0 & 0 & 1 & 1 & 0 & 1
\end{bmatrix}$, $b=\begin{bmatrix}
50\\60\\70\\80
\end{bmatrix}$.

It is easy to check that $A$ has full rank.
\subsection*{Part c}
The augmented matrix is
\[
A=\begin{bmatrix}
1 & 0 & 1 & 0 & 0 & 0 & | & 50\\
0 & 1 & 0 & 1 & 0 & 0 & | & 60\\
1 & 1 & 0 & 0 & 1 & 0 & | & 70\\
0 & 0 & 1 & 1 & 0 & 1 & | & 80
\end{bmatrix}.
\]
Notice that the matrix formed by the first 4 columns has rank 3 so we can swap
columns 4 and 6 to get
\[
\begin{bmatrix}
1 & 0 & 1 & 0 & 0 & 0 & | & 50\\
0 & 1 & 0 & 0 & 0 & 1 & | & 60\\
1 & 1 & 0 & 0 & 1 & 0 & | & 70\\
0 & 0 & 1 & 1 & 0 & 1 & | & 80
\end{bmatrix}
\]
and now the matrix formed by the first 4 columns has rank 4, then row reducing
we obtain
\[
\begin{bmatrix}
1 & 0 & 0 & 0 &  1 & -1 & | & 10\\
0 & 1 & 0 & 0 &  0 &  1 & | & 60\\
0 & 0 & 1 & 0 & -1 &  1 & | & 40\\
0 & 0 & 0 & 1 &  1 &  0 & | & 40
\end{bmatrix},
\]
which gives us $[x_1,x_2,x_3,x_6,x_5,x_4]=[10,60,40,40,0,0]$.
\section*{Problem 3}
\end{document}
