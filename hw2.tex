\documentclass{article}
\usepackage{amsmath}
\usepackage{amssymb}
\usepackage{graphicx}
\newcommand{\bld}[1]{\boldsymbol{#1}}

\begin{document}

\title{MATH520 Homework 1}
\author{Fernando}
\date{\today}
\maketitle

\section*{Exercise 6.3}
Show that if $\bld{x}^*$ is a global minimizer of $f$ over $\Omega$, and
$\bld{x}^*\in\Omega'\subset \Omega$, then $\bld{x}^*$ is a global minimizer of $f$ over
$\Omega'$.

\textbf{Solution:}

By definition: $f(\bld{x}^*)\leq f(\bld{x})$ for all $\bld{x}\in \Omega$; in
particular, it is true for all $\bld{x}\in\Omega'$ (because $\Omega'\subset
\Omega$). Then again by definition $\bld{x}^*$ is a global minimizer of $f$
over $\Omega'$.
\section*{Exercise 6.8}
Consider the following function $f:\mathbb{R}^2 \to \mathbb{R}$:
\[
	f(\bld{x})=\bld{x}^T
	\begin{bmatrix}
	1 & 2\\
	4 & 7
	\end{bmatrix}
	\bld{x} + \bld{x}^T
	\begin{bmatrix}
	3\\
	5
	\end{bmatrix}
	+ 6.
\]
a. Find the gradient and Hessian of $f$ at the point $[1,1]^T$.\\
b. Find the directional derivative of $f$ at $[1,1]^T$ with respect to a unit
vector in the direction of maximal rate of increase.\\
c. Find a point that satisfies the FONC (interior case) for $f$. Does this
point satisfy the SONC (for a minimizer)?

\textbf{Solution:}

Part a. Let
\[
	A = \begin{bmatrix}
		1 & 2\\
		4 & 7
	\end{bmatrix},
\]
then using the identity (found in the book)
$D(\bld{x}^TA\bld{x})=\bld{x}^T(A+A^T)$ (when $A$ is squared) we get
\begin{align*}
\nabla f(\bld{x}) &= \left[\bld{x}^T(A+A^T)\right]^T + [3,5]^T\\
&= \begin{bmatrix}
	2 & 6\\
	6 & 14
\end{bmatrix}\bld{x} + \begin{bmatrix}
3\\
5
\end{bmatrix}\\
&=[2x_1+6x_2+3,6x_1+14x_2+5]^T.
\end{align*}
Then at $[1,1]^T$ we get
\[
\nabla f([1,1]^T)=[11,25]^T.
\]

As for the Hessian, using the previuos calculation we get:
\[
\begin{bmatrix}
\partial_{x_1}\partial_{x_1}f & \partial_{x_2}\partial_{x_1}f\\
\partial_{x_1}\partial_{x_2}f & \partial_{x_2}\partial_{x_2}f
\end{bmatrix}
=
\begin{bmatrix}
2 & 6\\
6 & 14
\end{bmatrix}.
\]
This is the Hessian at any point, in particular at $[1,1]^T$.

Part b. Because this function is smooth we can compute the directional
derivative as
\[
	\frac{[11,25]^T\cdot
	[11,25]^T}{||[11,25]^T||}=||[11,25]^T||=\sqrt{11^2+25^2}.
\]

Part c. For this we need to solve the system
\[
	\begin{cases}
	2x_1+6x_2+3=0\\
	6x_1+14x_2+5=0
	\end{cases}
\]
Which gives us $x_1=3/2$ and $x_2=-1$ so the point $[3/2,-1]^T$ satisfies the
FONC for $f$. Using Sylvester's criterion we get that the matrix is indefinite
($\Delta_1=2, \Delta_2=-8$)
so it does not satisfy the SONC.
\section*{Exercise 6.19}
An art collector stands at a distance of $x$ feet from the wall, there a piece
of art (picture) of height $a$ feet is hung, $b$ feet above his eyes, as shown
in figure $\ref{fig619}$. Find the distance from the wall for which the angle $\theta$
subtended by the eye to the picture is maximized.\\
\begin{figure}[ht]
	\center
	\caption{Figure for exercise 6.19}
	\includegraphics[width=0.5\textwidth]{prob-6.19.png}
	\label{fig619}
\end{figure}

\textbf{Solution:}

Let $\theta_1$ be the angle on the eye vertex of the biggest triangle and
$\theta_2$ be the angle on the eye vertex of the lower triangle. Then
$\theta=\theta_1 - \theta_2$. As suggested by the book we can optimize
$\tan(\theta)$ instead of $\theta$ because tangent is increasing and
$\theta\in[0,\pi/2]$ (because of the physical limitations of the problem).
Using the second suggestion from the book we have to optimize
\[
	(\tan(\theta_1) - \tan(\theta_2))/(1+\tan(\theta_1)\tan(\theta_2))=\frac{a}{x+b(a+b)/x}.
\]
Taking the derivative we get
\[
\frac{ab(a+b)-ax^2}{(b(a+b)+x^2)^2}
\]
So the candidate points are $x=\sqrt{b(a+b)}$. Notice that the negative is not
considered because it doesn't make sense. Also it is a maximum because we can
see that physically the problem creates a function that is concave.
\section*{Exercise 6.29}
Line fitting. Let $[x_1,y_1]^T,\dots,[x_n,y_n]^T$, $n\geq 2$, be points on the
$\mathbb{R}^2$ plane (each $x_i,y_i\in \mathbb{R}$). We wish to find the
straight line of the "best fit" through these points ("best" in the sense that
the average squared error is minimized); that is, we wish to find $a,b\in
\mathbb{R}$ to minimize
\[
	f(a,b)=\frac{1}{n} \sum_{i=1}^n(ax_i+b-y_i)^2.
\]
a. Let
\begin{align*}
	\overline{X}=\frac{1}{n}\sum_{i=1}^n x_i.\\
	\overline{Y}=\frac{1}{n}\sum_{i=1}^n y_i.\\
	\overline{X^2}=\frac{1}{n}\sum_{i=1}^n x_i^2.\\
	\overline{Y^2}=\frac{1}{n}\sum_{i=1}^n y_i^2.\\
	\overline{XY}=\frac{1}{n}\sum_{i=1}^n x_iy_i.\\
\end{align*}
Show that $f(a,b)$ can be written in the form $\bld{z}^TQ\bld{z} -
2\bld{c}^T\bld{z} + d$, where $\bld{z}=[a,b]^T$, $Q=Q^T\in \mathbb{R}^{2\times
2}$, $\bld{c}\in \mathbb{R}^2$ and $d\in\mathbb{R}$, and find expressions for
$\bld{Q},\bld{c}$, and $d$ in terms of $\overline{X}, \overline{Y},
\overline{X^2}, \overline{Y^2}$, and $\overline{XY}$.\\
b. Assume that the $x_i, i=1,\dots,n$, are not all equal. Find the parameters
$a^*$ and $b^*$ for the line of best fit in terms of
$\overline{X},\overline{Y},\overline{X^2},\overline{Y^2}$ and $\overline{XY}$.
Show that the point $[a^*,b^*]^T$ is the only local minimizer of $f$.\\
c. Show that if $a^*$ and $b^*$ are the parameters of the line of best fit,
then $\overline{Y}=a^*\overline{X}+b^*$ (and hence once we have computed $a^*$,
we can compute $b^*$ using the formula $b^*=\overline{Y}-a^*\overline{X}$).

\textbf{Solution:}

a. We begin by noticing that
\[
(ax_i+b-y_i)^2=a^2x_i^2+b^2+y_i^2+2bax_i-2ax_iy_i-2by_i.
\]
Taking a sum over all $i$ and multiplying by $\frac{1}{n}$ we get that
\[
f(a,b) = a^2\overline{X^2}+ \overline{Y^2}-
2b\overline{Y}+2ab\overline{X} - 2a\overline{XY} + b^2.
\]
If we compute the quadratic form (remembering that $Q$ is symmetric) we get
\[
\bld{z}^TQ\bld{z} - 2\bld{c}^T\bld{z} + d =
a^2q_{11} + 2abq_{12}+b^2q_{22}-2ac_1 -2bc_2 +d
\]
Then by comparison we can set
\begin{align*}
Q&=\begin{bmatrix}
	\overline{X^2} & \overline{X}\\
	\overline{X} & 1
\end{bmatrix}\\
\bld{c}&=[\overline{XY},\overline{Y}]^T\\
d&=\overline{Y^2}
\end{align*}

b. First
\[
	\nabla f(z)^T=2z^TQ-2\bld{c}^T.
\]
Setting this equal to 0 we obtain:
\[
	z^TQ=\bld{c}^T,
\]
so
\[
	(z^*)^T=\bld{c}^TQ^{-1},
\]
or equivalently (remembering that $(A^{-1})^T=(A^T)^{-1}$)
\[
	z^*=Q^{-1}\bld{c}.
\]

To verify that this is a minimizer we compute
\[
	D^2f(z)=2Q
	=
	2\begin{bmatrix}
		\overline{X^2} & \overline{X}\\
		\overline{X} & 1
	\end{bmatrix}
\]
It is enough to prove that this matrix is positive definite.
Using Sylvester's criterion we get $\Delta_1=\overline{X^2}>0$ ($x_i$'s are not
equal so at most one is 0, we are assuming $i>1$). Then
$\Delta_2=\overline{X^2}-\overline{X}^2$, which according to the book is
$(1/n)\sum_{i=1}^n(x_i-\overline{X})^2$, so it is a strictly positive number
because again all the $x_i$'s are different, Hence $\Delta_2>0$ and by
Sylvester's criterion the matrix is positive definite.

c.
\end{document}
